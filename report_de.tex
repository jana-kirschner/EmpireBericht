\documentclass[de]{agse-empir-report}\usepackage[]{graphicx}\usepackage[]{color}
%% maxwidth is the original width if it is less than linewidth
%% otherwise use linewidth (to make sure the graphics do not exceed the margin)
\makeatletter
\def\maxwidth{ %
  \ifdim\Gin@nat@width>\linewidth
    \linewidth
  \else
    \Gin@nat@width
  \fi
}
\makeatother

\definecolor{fgcolor}{rgb}{0.345, 0.345, 0.345}
\newcommand{\hlnum}[1]{\textcolor[rgb]{0.686,0.059,0.569}{#1}}%
\newcommand{\hlstr}[1]{\textcolor[rgb]{0.192,0.494,0.8}{#1}}%
\newcommand{\hlcom}[1]{\textcolor[rgb]{0.678,0.584,0.686}{\textit{#1}}}%
\newcommand{\hlopt}[1]{\textcolor[rgb]{0,0,0}{#1}}%
\newcommand{\hlstd}[1]{\textcolor[rgb]{0.345,0.345,0.345}{#1}}%
\newcommand{\hlkwa}[1]{\textcolor[rgb]{0.161,0.373,0.58}{\textbf{#1}}}%
\newcommand{\hlkwb}[1]{\textcolor[rgb]{0.69,0.353,0.396}{#1}}%
\newcommand{\hlkwc}[1]{\textcolor[rgb]{0.333,0.667,0.333}{#1}}%
\newcommand{\hlkwd}[1]{\textcolor[rgb]{0.737,0.353,0.396}{\textbf{#1}}}%
\let\hlipl\hlkwb

\usepackage{framed}
\makeatletter
\newenvironment{kframe}{%
 \def\at@end@of@kframe{}%
 \ifinner\ifhmode%
  \def\at@end@of@kframe{\end{minipage}}%
  \begin{minipage}{\columnwidth}%
 \fi\fi%
 \def\FrameCommand##1{\hskip\@totalleftmargin \hskip-\fboxsep
 \colorbox{shadecolor}{##1}\hskip-\fboxsep
     % There is no \\@totalrightmargin, so:
     \hskip-\linewidth \hskip-\@totalleftmargin \hskip\columnwidth}%
 \MakeFramed {\advance\hsize-\width
   \@totalleftmargin\z@ \linewidth\hsize
   \@setminipage}}%
 {\par\unskip\endMakeFramed%
 \at@end@of@kframe}
\makeatother

\definecolor{shadecolor}{rgb}{.97, .97, .97}
\definecolor{messagecolor}{rgb}{0, 0, 0}
\definecolor{warningcolor}{rgb}{1, 0, 1}
\definecolor{errorcolor}{rgb}{1, 0, 0}
\newenvironment{knitrout}{}{} % an empty environment to be redefined in TeX

\usepackage{alltt}

% Blind texts, for demonstration only, not part of the actual template
\usepackage{lipsum}

% Load R code


\IfFileExists{upquote.sty}{\usepackage{upquote}}{}
\begin{document}

\title{Inhaltlich sinnfreier, aber demonstrativer Titel für die
    \LaTeX-Vorlage: Ein Beispieldokument}
\author{
    \authorName{Franz Zieris}
    \authorMail{zieris@inf.fu-berlin.de}
    \and
    \authorName{Hugo Schupps}
    \authorMail{schlupps@example.org}
    \and
    \authorName{Peter Bunse}
    \authorMail{bunsen@brenner.com}
    \and
    \authorName{Karla Kolumna}
    \authorMail{presse@aol.de}
    \and
    \authorName{Margot Delia Schröter}
    \authorMail{m.d.schroeter@t-online.de}
}

\maketitle

% Load R chunk


\begin{abstract}
    \lipsum[1]
    Insgesamt haben an der Studie 79 Personen
    aus 27 Ländern teilgenommen.
\end{abstract}


\section[fz]{Einführung}

Explanation of the context.
Research question: description, justification of relevance,
and motivation.
State and explain your hypotheses.
Give an overview of the structure of the rest of the article.

\lipsum[2]


\section[kk]{Verwandte Arbeiten}

Discussion of other possible research questions and other possible
empirical methods for this area of interest.
Mention and cite related studies in this area.

\lipsum[3]


\section[hs]{Methode}

Description of the most important considerations for the
questionnaire's design (incl. discussion of problems):
Formulation and order of questions, scale types, relevance as to the
hypotheses.
Short description of the recruitment method for participants,
including a characterization of your target group.

\lipsum[4]


\section[pb]{Datenanalyse \& Resultate}

\begin{figure*}
    \includegraphics[width=\linewidth]{creative_commons.jpg}
    \caption{Das Creative-Commons-Logo als breite Grafik}
\end{figure*}

Number and characterization of respondents.
Description of the approach for the data validation and
analysis, short explanation of important scripts you used.
Description of the considerations and the results of your search for
scientific statements and correlations; possibly with quantitative
results and/or graphic visualizations.

\lipsum[5]


\subsection{Dieses}
\lipsum[6]


\subsection[mds]{Jenes}
\lipsum[7-8]


\section[hs]{Schlussfolgerungen}
Summary of the most important insights from the analysis and
answer to the research question with respect to your hypotheses.
If answering your research question is not possible, discuss why.
Discussion of the threats to validity and the survey's
shortcomings as well as evaluation of credibility and relevance.

\lipsum[9-10]


\section[kk]{Reflektion}

\begin{figure}
    \includegraphics[width=\linewidth]{creative_commons.jpg}
    \caption{Das Creative-Commons-Logo als Spaltengrafik}
\end{figure}

What did you learn from (or became aware of during) this project with
respect to: choice and formulation of a research question,
drafting and implementation of a questionnaire,
recruitment of participants,
data collection, evaluation, and drawing of conclusions?
Evaluate your approach in view of the general approach for
empiricism (see
\url{http://www.inf.fu-berlin.de/inst/ag-se/teaching/V-EMPIR-2017/11_generic_method.pdf}).

\lipsum[11-13]

\begin{figure}
\begin{knitrout}
\definecolor{shadecolor}{rgb}{0.969, 0.969, 0.969}\color{fgcolor}\begin{kframe}


{\ttfamily\noindent\bfseries\color{errorcolor}{\#\# Error in library("{}ggplot2"{}): there is no package called 'ggplot2'}}

{\ttfamily\noindent\bfseries\color{errorcolor}{\#\# Error in eval(expr, envir, enclos): konnte Funktion "{}ggplot"{} nicht finden}}

{\ttfamily\noindent\bfseries\color{errorcolor}{\#\# Error in print(plt): Objekt 'plt' nicht gefunden}}\end{kframe}
\end{knitrout}
    \caption{Ein kleiner R-Plot}
\end{figure}

\begin{table}
    \centering
\begin{knitrout}
\definecolor{shadecolor}{rgb}{0.969, 0.969, 0.969}\color{fgcolor}
\begin{tabular}{rrrrrll}
\toprule
x & y & sum & ratio & diff & sum\_big & same\\
\midrule
2 & 11 & 13 & 0.18 & 9 & no & no\\
9 & 10 & 19 & 0.90 & 1 & no & no\\
16 & 9 & 25 & 1.78 & 7 & yes & no\\
6 & 12 & 18 & 0.50 & 6 & no & no\\
11 & 7 & 18 & 1.57 & 4 & no & no\\
\bottomrule
\end{tabular}


\end{knitrout}
    \caption{Eine kleine Tabelle}
\end{table}

% For printing all bib entries; remove to only print actually cited entries
\nocite{*}

\bibliography{sample}


\appendix

\section{Anschreiben}

...


\section{Fragebogen}

...


\section{Rohdaten und Auswertungskripte}

URL zum Download der Rohdaten und der R-Analyseskripte, die sich direkt auf den
Rohdaten ausführen lassen.
Idealerweise ein Git-Repository.

\end{document}
